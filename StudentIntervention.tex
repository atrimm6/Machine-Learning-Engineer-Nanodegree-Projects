\documentclass[12 pt]{article}
\usepackage{amssymb}
\usepackage{amsmath}
\usepackage{amsthm}
\usepackage{amsfonts}
\usepackage{fullpage}
\usepackage[mathscr]{euscript}
\usepackage{youngtab}
\usepackage{graphicx}
\usepackage{color}
\usepackage{multirow}
\usepackage{enumerate}
\usepackage{listings}
\newcommand{\W}{\mathcal{W}}
\newcommand{\C}{\mathbb{C}}
\newcommand{\R}{\mathbb{R}}
\newcommand{\Z}{\mathbb{Z}}
\newcommand{\N}{\mathcal{N}}
\newcommand{\M}{\mathcal{M}}
\newcommand{\OO}{\mathcal{O}}
\newcommand{\B}{\mathcal{B}}
\newcommand{\pf}{\tilde{\phi}_N}
\newcommand{\g}{\mathfrak{g}}
\newcommand{\su}{\mathfrak{su}}
\newcommand{\so}{\mathfrak{so}}
\newcommand{\usp}{\mathfrak{usp}}
\newcommand{\h}{\mathfrak{h}}
\newcommand{\BB}{\mathfrak{B}}
\newcommand{\mon}{\OO_{\vec{n}}(a)}
\newcommand{\I}{\mathcal{I}}
\numberwithin{equation}{section}
\begin{document}

\title{Student Intervention System \\ \footnotesize{Udacity Machine Learning Engineer \\ Nanodegree Program: Project 2}}
\author{Anderson Daniel Trimm}
\date{\today}
\maketitle

\section{Classification vs Regression}
Our model predicts which students will pass or fail their high school final exam, and therefore need an intervention. Since we are predicting \emph{discrete} labels (in this case, binary), this is a \emph{classification} problem, as opposed to a \emph{regression} problem, in which we would be predicting \emph{continuous} labels.
\section{Exploring the Data}
In this section, we use the code in the accompanying ipython notebook to identify important characteristics of the dataset which will influence our prediction. Running the code, we find

\begin{verbatim}	
Total number of students: 395
Number of students who passed: 265
Number of students who failed: 130
Number of features: 31
Graduation rate of the class: 0.67%
\end{verbatim}

\section{Training and Evaluating the Models}
\subsection{KNN Classifier}
The $k$-nearest neighbor classifier finds the $k$ training samples closest in distance to the query point, and predicts the label by comparing to these. \\
\\
Pros:
\begin{itemize}
	\item Fast training time - it simply remembers all the training points
	\item Being non-parametric, it can be useful in classification problems where the decision boundary is very irregular
\end{itemize}

Cons:
\begin{itemize}
	\item Potentially long training time - it has to search through the dataset to find the $k$ nearest neighbors
	\item Uses a lot of CPU memory, since it has to store the dataset (as opposed to a parametric model, which throws away the dataset after learning the parameters)
\end{itemize}

In our case, we don't have a good hypothesis for a parametric model, so a non-parametric model might be best. Despite the memory cost, since our dataset is not too large, kNN is a reasonable classifier to try. The results for our model with the default parameter settings are recorded in the table below.

\begin{center}
  \begin{tabular}{| c | c | c | c | }
    \hline
    & 100 & 200 & 300 \\ \hline
    Training time (sec) & 0.001 & 0.001 & 0.001 \\ \hline
    Prediction time (sec) & 0.001 & 0.002 & 0.003 \\ \hline 
    F1 score for training set & 0.88  & 0.88 & 0.88 \\ \hline
    F1 score for test set & 0.78 & 0.77 & 0.78 \\ \hline
  \end{tabular}
\end{center}

\subsection{Gaussian Naive Bayes Classifier}
Naive Bayes is a learning algorithm based on applying Bayes' theorem with the assumption that every pair of features are independent (hence the use of the word ``naive"). Here, we use the Gaussian naive Bayes classifier, which assumes the likelihood of each feature is Gaussian.
Pros:
\begin{itemize}
	\item Querying is fast
	\item Few parameters (linear in number of variables)
	\item Performs well on big feature spaces
\end{itemize}

Cons:
\begin{itemize}
	\item Learning is slow - have to learn all the joint probabilities
	\item Doesn't model interrelations between attributes
\end{itemize}

In contrast with KNN, naive Bayes is a parametric model, so while learning is slower, querying is faster. It therefore makes sense to compare the performance of this kind of model with that of an instance based model, such as  KNN. The results are recorded in the table below.

\begin{center}
  \begin{tabular}{| c | c | c | c | }
    \hline
    & 100 & 200 & 300 \\ \hline
    Training time (sec) & 0.001 & 0.001 & 0.001 \\ \hline
    Prediction time (sec) & 0.001 & 0.000 & 0.000 \\ \hline 
    F1 score for training set & 0.33 & 0.83 & 0.80 \\ \hline
    F1 score for test set & 0.24 & 0.74 & 0.76\\ \hline
  \end{tabular}
\end{center}

\subsection{Support Vector Machine Classifier}
Pros:
\begin{itemize}
	\item Effective in high dimensional spaces
	\item Can still be effective in cases where the number of dimensions is greater than the number of samples
	\item Uses only a subset of training points in the decision function (the ``support vectors"), so it is memory efficient
	\item Versatile, since different kernel functions can be specified for the decision function
\end{itemize}
Cons:
\begin{itemize}
	\item If the number of features is much greater than the number of samples, it is likely to perform poorly
	\item Does not directly provide probability estimates; these are calculated using a memory expensive five-fold cross-validation
\end{itemize}

Since SVM only uses a subset of the training points in its decision function, it should be more memory efficient than KNN. However, since it is a parametric model, it will take longer to train the model. It is therefore worth investigating this tradeoff by comparing its performance against that of KNN.

\begin{center}
  \begin{tabular}{| c | c | c | c | }
    \hline
    & 100 & 200 & 300 \\ \hline
    Training time (sec) & 0.001 & 0.004 & 0.008 \\ \hline
    Prediction time (sec) & 0.001 & 0.002 & 0.002 \\ \hline 
    F1 score for training set & 0.90 & 0.89 & 0.87 \\ \hline
    F1 score for test set & 0.78 & 0.80 & 0.78 \\ \hline
  \end{tabular}
\end{center}

\section{Choosing the Best Model}
\subsection{Optimal Model}
Comparing the three tables, we see that both KNN and SVM outperformed Naive Bayes in terms of F1 score as the size of the training set decreased. In fact, their F1 score did not decrease by decreasing the size of the training set from 300 to 100. We can therefore restrict our training set to only 100 data points, which leads to savings on memory cost as well as computation time. Between these two algorithms, support vector machines still uses less memory as it is parametric. Since the training and prediction times, as well as the F1 score on the test set for these two algorithms are the same (training on 100 data points), we choose SVM as the optimal model.
\subsection{Description of Model in Layman's Terms}
To determine which students will pass or fail using our data, we first plot the data. We will pretend there are only two features, so that we can plot these points in the xy-plane on a sheet of paper. The general idea will be the same. Our model (SVM) will determine where to draw a line that separates the xy-plane into two regions: data points falling into one region will all be passing, and all those falling into the other will be failing. This line will be chosen as to minimize the number of misclassifications, while at the same time maximizing its ability to correctly classify new, unseen data points.

While many models create this kind of linear boundary, the advantage of our model over others is that it chooses the line that maximizes the distance between the two classes of data points (passing or failing). It is this maximum distance that allows our model to generalize well to new data.

In some situations, the data points might be mixed up in such a way that it may not be possible to find a line to separate the passing and failing students. If this is the case, it is no problem, as our model can still separate them with the following modification: we add a third axis (a ``depth", that would be coming out of the page) and plot the data points in this three-dimensional space. By making the ``depth" of the data point depend on their positions in the xy-plane, we can find a plane that separates the passing and failing students (in three-dimensions). This plane is again chosen so that there is a maximum distance between the passing and failing students, so that the model is able to generalize well to new data.

\subsection{Model Tuning and Tuned F1 Score}
We tune our SVM model using gridsearch over the parameter ``C", which is the penalty parameter of the error term. Running the code, we find the optimal C-value is 0.5. The optimized model gives has an F1 score of 0.82 on the test set, which is indeed an improvement of the F1 score of 0.78 of the default SVM model.
\end{document}